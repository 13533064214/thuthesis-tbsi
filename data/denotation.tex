% !TeX root = ../thuthesis-example.tex

\begin{denotation}[3cm]
  \item[${\rm \mathbf{A}}$] Matrix
  \item[$\mathbf{a}$] Vector
  \item[$a$] Scalar
  \item[${\rm \mathbf{W}}$] Weight matrix
  \item[$\omega$] A set of random variables $\{{\rm \mathbf{W}}_1, \dots, {\rm \mathbf{W}}_L\}$
  \item[$f^\omega$] Function parametrized by the variables $\omega$
  \item[$S$] Dataset
  \item[${\rm \mathbf{X}}$] Dataset inputs (matrix with $N$ rows, one for each sample)
  \item[${\rm \mathbf{Y}}$] Dataset outputs (matrix with $N$ rows, one for each sample)
  \item[$\mathbf{x}_i$] Input sample for model
  \item[$\mathbf{y}_i$] Output label for model
  \item[$\mathbf{z}_i$] Data point combining both input and output $(\mathbf{x}_i,\mathbf{y}_i)$
  \item[$\hat{\mathbf{y}}_i$] Model prediction on input sample $\mathbf{x}_i$
  \item[$\ell$] Loss function
  \item[$\mathcal{N}$] The Gaussian distribution
  \item[$\mathbb{R}$] The real numbers
  \item[IB] Information bottleneck
  \item[BNN] Bayesian neural network
  \item[VI] Variational inference
  \item[ELBO] Evidence lower bound
  \item[KL] Kullback–Leibler
  \item[DNN] Deep neural network
  \item[MC] Monte Carlo
  \item[MCMC] Markov chain Monte Carlo
  \item[MDL] Minimum description length
  \item[PAC] Probably approximately correct 
  \item[e.g.] Exempli gratia (“for the sake of an example”)
  \item[i.e.] Id est (“it is”)
  \item[i.i.d.] Independent and identically distributed
  \item[s.t.] Subject to
  \item[w.r.t.] With respect to
\end{denotation}



% \printnomenclature[3cm]
% asgsaf

% 也可以使用 nomencl 宏包,需要在导言区
% \usepackage{nomencl}
% \makenomenclature

% 在这里输出符号说明
% \printnomenclature[3cm]

% 在正文中的任意为都可以标题
% \nomenclature{PI}{聚酰亚胺}
% \nomenclature{MPI}{聚酰亚胺模型化合物,N-苯基邻苯酰亚胺}
% \nomenclature{PBI}{聚苯并咪唑}
% \nomenclature{MPBI}{聚苯并咪唑模型化合物,N-苯基苯并咪唑}
% \nomenclature{PY}{聚吡咙}
% \nomenclature{PMDA-BDA}{均苯四酸二酐与联苯四胺合成的聚吡咙薄膜}
% \nomenclature{MPY}{聚吡咙模型化合物}
% \nomenclature{As-PPT}{聚苯基不对称三嗪}
% \nomenclature{MAsPPT}{聚苯基不对称三嗪单模型化合物,3,5,6-三苯基-1,2,4-三嗪}
% \nomenclature{DMAsPPT}{聚苯基不对称三嗪双模型化合物(水解实验模型化合物)}
% \nomenclature{S-PPT}{聚苯基对称三嗪}
% \nomenclature{MSPPT}{聚苯基对称三嗪模型化合物,2,4,6-三苯基-1,3,5-三嗪}
% \nomenclature{PPQ}{聚苯基喹噁啉}
% \nomenclature{MPPQ}{聚苯基喹噁啉模型化合物,3,4-二苯基苯并二嗪}
% \nomenclature{HMPI}{聚酰亚胺模型化合物的质子化产物}
% \nomenclature{HMPY}{聚吡咙模型化合物的质子化产物}
% \nomenclature{HMPBI}{聚苯并咪唑模型化合物的质子化产物}
% \nomenclature{HMAsPPT}{聚苯基不对称三嗪模型化合物的质子化产物}
% \nomenclature{HMSPPT}{聚苯基对称三嗪模型化合物的质子化产物}
% \nomenclature{HMPPQ}{聚苯基喹噁啉模型化合物的质子化产物}
% \nomenclature{PDT}{热分解温度}
% \nomenclature{HPLC}{高效液相色谱 (High Performance Liquid Chromatography)}
% \nomenclature{HPCE}{高效毛细管电泳色谱 (High Performance Capillary lectrophoresis)}
% \nomenclature{LC-MS}{液相色谱-质谱联用 (Liquid chromatography-Mass Spectrum)}
% \nomenclature{TIC}{总离子浓度 (Total Ion Content)}
% \nomenclature{\textit{ab initio}}{基于第一原理的量子化学计算方法,常称从头算法}
% \nomenclature{DFT}{密度泛函理论 (Density Functional Theory)}
% \nomenclature{$E_a$}{化学反应的活化能 (Activation Energy)}
% \nomenclature{ZPE}{零点振动能 (Zero Vibration Energy)}
% \nomenclature{PES}{势能面 (Potential Energy Surface)}
% \nomenclature{TS}{过渡态 (Transition State)}
% \nomenclature{TST}{过渡态理论 (Transition State Theory)}
% \nomenclature{$\increment G^\neq$}{活化自由能(Activation Free Energy)}
% \nomenclature{$\kappa$}{传输系数 (Transmission Coefficient)}
% \nomenclature{IRC}{内禀反应坐标 (Intrinsic Reaction Coordinates)}
% \nomenclature{$\nu_i$}{虚频 (Imaginary Frequency)}
% \nomenclature{ONIOM}{分层算法 (Our own N-layered Integrated molecular Orbital and molecular Mechanics)}
% \nomenclature{SCF}{自洽场 (Self-Consistent Field)}
% \nomenclature{SCRF}{自洽反应场 (Self-Consistent Reaction Field)}
