% !TeX root = ../thuthesis-example.tex

\begin{resume}

  \section*{个人简历}

197×年××月××日出生于四川××县。

1992年9月考入××大学化学系××化学专业,1996年7月本科毕业并获得理学学士学位。

1996年9月免试进入清华大学化学系攻读××化学博士至今。

  \section*{在学期间完成的相关学术成果}

  \subsection*{学术论文}

  \begin{achievements}
    \item \textbf{Yang Y}, Ren T L, Zhang L T, et al. Miniature microphone with silicon- based ferroelectric thin films. Integrated Ferroelectrics, 2003, 52:229-235. (SCI收录, 检索号:758FZ.)
    \item \textbf{杨轶}, 张宁欣, 任天令, 等. 硅基铁电微声学器件中薄膜残余应力的研究. 中国机械工程, 2005, 16(14):1289-1291. (EI收录, 检索号:0534931 2907.)
    \item \textbf{杨轶}, 张宁欣, 任天令, 等. 集成铁电器件中的关键工艺研究. 仪器仪表学报, 2003, 24(S4):192-193. (EI源刊.)
    \item \textbf{Yang Y}, Ren T L, Zhu Y P, et al. PMUTs for handwriting recognition. In press. (已被Integrated Ferroelectrics录用. SCI源刊.)
    \item Wu X M, \textbf{Yang Y}, Cai J, et al. Measurements of ferroelectric MEMS microphones. Integrated Ferroelectrics, 2005, 69:417-429. (SCI收录, 检索号:896KM.)
    \item 贾泽, \textbf{杨轶}, 陈兢, 等. 用于压电和电容微麦克风的体硅腐蚀相关研究. 压电与声光, 2006, 28(1):117-119. (EI收录, 检索号:06129773469.)
    \item 伍晓明, \textbf{杨轶}, 张宁欣, 等. 基于MEMS技术的集成铁电硅微麦克风. 中国集成电路, 2003, 53:59-61.
  \end{achievements}


%   \subsection*{专利}

%   \begin{achievements}
%     \item 任天令, 杨轶, 朱一平, 等. 硅基铁电微声学传感器畴极化区域控制和电极连接的方法: 中国, CN1602118A[P]. 2005-03-30.
%     \item Ren T L, Yang Y, Zhu Y P, et al. Piezoelectric micro acoustic sensor based on ferroelectric materials: USA, No.11/215, 102[P]. (美国发明专利申请号.)
%   \end{achievements}

\end{resume}
